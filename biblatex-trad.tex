\documentclass{ltxdockit}[2011/03/25]
\usepackage{btxdockit}
\usepackage[utf8]{inputenc}
\usepackage[american]{babel}
\usepackage[strict]{csquotes}
\usepackage{shortvrb}
\usepackage{pifont}
\usepackage{libertine}
\usepackage[scaled=0.8]{beramono}
\usepackage{microtype}
\lstset{basicstyle=\ttfamily,keepspaces=true}
\KOMAoptions{numbers=noenddot}
\addtokomafont{paragraph}{\spotcolor}
\addtokomafont{section}{\spotcolor}
\addtokomafont{subsection}{\spotcolor}
\addtokomafont{subsubsection}{\spotcolor}
\addtokomafont{descriptionlabel}{\spotcolor}
\pretocmd{\cmd}{\sloppy}{}{}
\pretocmd{\bibfield}{\sloppy}{}{}
\pretocmd{\bibtype}{\sloppy}{}{}


\MakeAutoQuote*{<}{>}
\MakeShortVerb{\|}

\newcommand*{\biber}{Biber\xspace}

\titlepage{%
  title={The \sty{biblatex-trad} Package},
  subtitle={Traditional bibliography styles for \sty{biblatex}},
  url={},
  author={Marco Daniel},
  email={},
  revision={0.1},
  date={09/2012}}

\hypersetup{%
  pdftitle={The biblatex-trad Package},
  pdfsubject={Traditional bibliography styles for biblatex},
  pdfauthor={Marco Daniel},
  pdfkeywords={tex, latex, bibtex, bibliography, references, citation}}


\newrobustcmd*{\Deprecated}{%
  \textcolor{spot}{\margnotefont Deprecated}}
\newrobustcmd*{\DeprecatedMark}{%
  \leavevmode\marginpar{\Deprecated}}
\newrobustcmd*{\BiberOnly}{%
  \textcolor{spot}{\margnotefont Biber only}}
\newrobustcmd*{\BiberOnlyMark}{%
  \leavevmode\marginpar{\BiberOnly}}
\newrobustcmd*{\BibTeXOnly}{%
  \textcolor{spot}{\margnotefont BibTeX only}}
\newrobustcmd*{\BibTeXOnlyMark}{%
  \leavevmode\marginpar{\BibTeXOnly}}

\hyphenation{%
  star-red
  bib-lio-gra-phy
  white-space
}

\begin{document}

\printtitlepage
\tableofcontents


\section{Introduction}\label{sec:int}

The package \sty{biblatex-trad} is a contribution to the great package \sty{biblatex}. 

It provides the implementation of the traditional bibliography styles (\sty{plain},
\sty{unsrt}, \sty{alpha} and \sty{abbrv}) as a style for \sty{biblatex}.

\subsection{Motivation}\label{subsec:int:mot}

The package is motivated by a question at \tex-\latex Stack Exchange \glqq \href{http://tex.stackexchange.com/}{How to emulate the traditional BibTeX styles (plain, abbrv, unsrt, alpha) as closely as possible with biblatex?}

\subsection{Requirements}

The usage of the styles requires \sty{biblatex} 2.0 or newer.

\subsection{License}

Copyright \textcopyright\ 2006--2012 Marco Daniel. Permission is granted to copy, distribute and\slash or modify this software under the terms of the \lppl, version 1.3.\fnurl{http://www.ctan.org/tex-archive/macros/latex/base/lppl.txt}.


\subsection{Feedback}\label{subsec:int:feb}

Please use the \sty{biblatex-trad} project page on GitHub to report bugs and submit feature requests.\fnurl{https://github.com/marcodaniel/trad-biblatex}

If you do not want to report a bug or request a feature but are simply in need of assistance, you might want to consider posting your question on the \texttt{comp.text.tex} newsgroup or \tex-\latex Stack Exchange.\fnurl{http://tex.stackexchange.com/questions/tagged/biblatex}

\section{Usage}

\sty{biblatex-trad} isn't a standalone package. As described in \secref{sec:int} it's
a small collection of styles. So you can load the styles as follows:

\begin{lstlisting}[style=latex]{}
\usepackage[style=XXX]{biblatex}
\end{lstlisting}
The available styles are listet below.
\begin{marglist}

\item[trad-plain] Implementation of the standard style \sty{plain}
\item[trad-unsrt] Implementation of the standard style \sty{unsrt}
\item[trad-alpha] \BiberOnlyMark  Implementation of the standard style \sty{alpha}
\item[trad-abbrv]  Implementation of the standard style \sty{abbrv}

\end{marglist}

After loading the style you can use all options provided by the package \sty{biblatex}.

\section{Limitation}

Up to know the package only setup the entry types \bibtype{BOOK}, \bibtype{ARTICLE} and
\bibtype{INCOLLECTION}.

\section{Revision History}
\label{apx:log}

\begin{changelog}

\begin{release}{0.1}{2012-09-07}
\item First upload
\end{release}

\end{changelog}
\end{document}

\endinput

